\documentclass[11pt]{article}
\usepackage[utf8]{inputenc}
\usepackage[spanish]{babel}
\usepackage[margin = 3.5cm]{geometry}
\usepackage{hyperref}

\title{Manual de Usuario}

\newcommand{\centro}{Centro de Microscop\'ia}

\newcommand{\correo}{\url{cmicroscopia@uniandes.edu.co}}

\begin{document}
	\maketitle
	\tableofcontents
	\newpage
	
	\section{Introducción}
	\section{Solicitud del servicio (usuarios nuevos)}
	Para el caso de usuarios que busquen usar los servicios del \centro\ por primera vez, es necesario realizar usa asesoria inicial. Este encuentro se realiza con el objetivo de determinar cu\'ales son las necesidades del usuario y c\'omo el \centro\ puede suplirlas.
	
	\section{Solicitud de cotizaci\'on}
	Para la elaboraci\'on de una cotizaci\'on son necesarios los siguientes datos:
		\begin{itemize}
			\item Nombre:
			\item Correo:
			\item Instituci\'on o empresa:
			\item Nit/C\'edula:
			\item Direcci\'on:
			\item Ciudad:
			\item Tel\'efono:
			\item El listado de los servicios y cantidades a usar. 
		\end{itemize}
	Los cuales deben ser enviados al correo \correo\ junto con la siguiente informaci\'on:
	
	\subsection{Usuarios internos}
	Para el caso de estudiantes (pregrado y posgrado) se debe contar con el apoyo de un profesor de planta de la universidad. El profesor asume los costos generados por el uso de los servicios del centro por parte del estudiante.
	
	En el caso que se haga parte de un proyecto de la universidad, es necesario adjuntar el nombre y c\'odigo del mismo.
	\subsection{Usuarios externos}
	Los usuarios externos deben especificar si el pago se realizar\'a como persona natural o persona jur\'idica. En el caso de ser persona natural se proceder\'a con la generaci\'on de un recibo. Las personas jur\'idicas deben enviar una orden de compra para generar una factura.
	
	\section{Manejo de cotizaci\'on}
	Antes de hacer uso de los servicios del \centro, es necesario que los usuarios anexen comprobante de pago al correo \correo. En el caso de los usuarios internos el comprobante consiste en la prueba de translado de fondos.
	
	\section{Preparaci\'on de la muestra}
	En ocasiones es posible que las muestras a analizar requieran de alg\'un tipo de preparaci\'on previa al uso de un equipo especializado. Lo anterior es particularmente importante para la microscop\'ia electr\'onica. Raz\'on por la cual en el \centro\ se prestan los siguientes servicios de preparaci\'on:
	\subsection{Fijaci\'on}
	\subsection{Deshidrataci\'on}
	\subsection{Metalizaci\'on}
	\subsection{Cortes}
	\subsubsection{Ultramicr\'otomo}
	\subsubsection{Disco de \'oxido de aluminio}
	
	\section{Reserva de equipos}
	\subsection{Microscop\'ia \'optica}
	\subsubsection{Confocal de barrido l\'aser}
	\subsubsection{Olympus}
	\subsection{Microscop\'ia de barrio por sonda}
	\subsection{Microscop\'ia de electr\'onica}
	
\end{document}